\documentclass[a4paper,12pt]{article}
\usepackage[slovene]{babel}
\usepackage[utf8]{inputenc}
\usepackage[T1]{fontenc}
\usepackage{lmodern}
\usepackage{graphicx}
\usepackage{amssymb}
\usepackage[top=1.3in, bottom=1.3in, left=1.25in, right=1.25in]{geometry}

\begin{document}
\parindent = 8mm
\begin{center}
\textbf{Reševanje problema trgovskega potnika s k-optimalnim in Lin-Kernighanovim algoritmom}\\
\vspace{5 mm}
\textbf{Žan Jernejčič in Ines Šilc}
\end{center}

V projektni nalogi bova reševala Problem trgovskega potnika s pomočjo k-optimalnega in Lin-Kernighanovega algoritma.\\

Problem trgovskega potnika oziroma Travelling salesman problem (krajše TSP) je problem, kjer imamo podanih $n$ mest in razdalje med vsemi (za vsak par mest imamo torej podano, koliko sta si oddaljeni). Zanima nas, ali lahko obiščemo vsako mesto in se na koncu vrnemo v prvotno mesto. Če označimo $d_{i, j}$ kot razdaljo mad $i$-tim in $j$-tim mestom, iščemo torej:

$$
\min_{\pi \in S_n} \sum\limits_{i=1}^{n-1} d_{\pi (i), \pi (i+1)} + d_{\pi (n), \pi (i)}
$$
 kjer je $S_n$ množica vseh permutacij danih $n$ mest.
 
Naivna rešitev je očitna, pogledamo $(n-1)!$ kombinacij, torej iz vsakega mesta v vsako drugo mesto, si zapišemo vse kombinacije in kakšno razdaljo smo prepotovali, ter izberemo tisto možnost, kjer je bila razdalja najkrajša. Poskusimo še z drugimi rešitvami.

\vspace{5 mm}
\textbf{K-optimalni algoritem}




\vspace{5 mm}
\textbf{Lin-Kernighanov algoritem}









\end{document}