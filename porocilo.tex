\documentclass[12pt, a4paper]{article}
\usepackage[utf8]{inputenc}
\usepackage[T1]{fontenc}
\usepackage[slovene]{babel}
\usepackage{amsmath}
\usepackage{eurosym}
\usepackage{hyperref}
\usepackage{graphicx}
\usepackage[top=2.5cm, bottom=2.5cm, left=3cm, right=2.5cm]{geometry}
\usepackage{indentfirst}
\setlength{\parindent}{0.5cm}

\begin{document}

\begin{titlepage}
\begin{center}

\large
Univerza v Ljubljani\\
\normalsize
Fakulteta za matematiko in fiziko\\

\vspace{3 cm} 

\large
Žan Jernejčič in Ines Šilc\\

\vspace{0.5cm}
\LARGE
\textbf{Reševanje problema trgovskega potnika s k-optimalnim in
Lin-Kernighanovim algoritmom}

\vfill

\large Ljubljana, 2020

\end{center}
\end{titlepage}

\newpage

\section[Definiranje problema]{Definiranje problema}

V projektni nalogi bova reševala Problem trgovskega potnika s pomočjo k-optimalnega in Lin-Kernighanovega algoritma.\\

Problem trgovskega potnika oziroma Travelling salesman problem (krajše TSP) je problem, kjer imamo podanih $n$ mest in razdalje med vsemi (za vsak par mest imamo torej podano, koliko sta si oddaljeni). Zanima nas, ali lahko obiščemo vsako mesto in se na koncu vrnemo v prvotno mesto. Če označimo $d_{i, j}$ kot razdaljo mad $i$-tim in $j$-tim mestom, iščemo torej:

$$
\min_{\pi \in S_n} \sum\limits_{i=1}^{n-1} d_{\pi (i), \pi (i+1)} + d_{\pi (n), \pi (1)}
$$
 kjer je $S_n$ množica vseh permutacij danih $n$ mest. \\
 
Naivna rešitev je očitna, pogledamo $(n-1)!$ kombinacij, torej iz vsakega mesta v vsako drugo mesto, si zapišemo vse kombinacije in kakšno razdaljo smo prepotovali, ter izberemo tisto možnost, kjer je bila razdalja najkrajša.\\ 

Pred začetkom se lahko vprašamo kakšen mora biti graf, da lahko na njem izvajamo sledeča algoritma. Graf \textbf{ne sme} imeti \textbf{negativnih ciklov}, saj je najcenejši cikel potem očiten, in ustvarimo neskončno zanko, saj bo imela najboljša rešitev ceno -$\infty$. \\

Za projekt sva uporabila programski jezik \texttt{python}. Uporabljala sva naslednje pakete:

\begin{itemize}

\item \texttt{networkx}: za definiranje in generiranje grafov

\item \texttt{}

\item \texttt{}

\item \texttt{}

\item \texttt{matplotlib}: za izrisovanje grafov

\item \texttt{time}: za mertive časovne zahtevnosti

\end{itemize}

Za preverjanje veljavnosti poti v grafu za problem potujočega trgovca, morava vedeti:

\begin{itemize}

\item Vsaka pot trgovca bo morala imeti vsa vozlišča zato vedno preverimo:
\begin{center}
\texttt{pot.order() == originalen\_graf.order()}
\end{center}
\item Vsaka pot trgovca bo morala imeti točno toliko povezav kot vozlišč:
\begin{center}
\texttt{pot.size(weight = None) == pot.order()} 
\end{center}
\item Dolžina poti: \texttt{pot.size()}

\end{itemize}











\newpage
\section[k-optimalni algoritmi]{k-optimalni algoritmi}

\subsection[2-opt algoritem]{2-opt algoritem}

\subsection[3-opt algoritem]{3-opt algoritem}

\newpage
\section[Lin-Kernighanov algoritem]{Lin-Kernighanov algoritem}

\newpage
\section[Viri]{Viri}

\end{document}